
\documentclass[a4paper, 12pt]{article} 

%--------------------------------------
%Russian-specific packages
%--------------------------------------
%\usepackage[warn]{mathtext}
\usepackage[T2A]{fontenc}
\usepackage[utf8]{inputenc}
\usepackage[english,russian]{babel}
\usepackage[intlimits]{amsmath}
\usepackage{esint}
%--------------------------------------
%Hyphenation rules
%--------------------------------------
\usepackage{hyphenat}
\hyphenation{ма-те-ма-ти-ка вос-ста-нав-ли-вать}
%--------------------------------------
%Packages
%--------------------------------------
\usepackage{amsmath}
\usepackage{amssymb}
\usepackage{amsfonts}
\usepackage{amsthm}
\usepackage{latexsym}
\usepackage{mathtools}
\usepackage{epstopdf}
\usepackage{etoolbox}%Булевые операторы
\usepackage{extsizes}%Выставление произвольного шрифта в \documentclass
\usepackage{geometry}%Разметка листа
\usepackage{indentfirst}
\usepackage{wrapfig}%Создание обтекаемых текстом объектов
\usepackage{fancyhdr}%Создание колонтитулов
\usepackage{setspace}%Настройка интерлиньяжа
\usepackage{lastpage}%Вывод номера последней страницы в документе, \lastpage
\usepackage{soul}%Изменение параметров начертания
\usepackage{hyperref}%Две строчки с настройкой гиперссылок внутри получаеммого
\usepackage[usenames,dvipsnames,svgnames,table,rgb]{xcolor}% pdf-документа
\usepackage{multicol}%Позволяет писать текст в несколько колонок
\usepackage{cite}%Работа с библиографией
\usepackage{subfigure}% Человеческая вставка нескольких картинок
\usepackage{tikz}%Рисование рисунков
\usepackage{float}
% Для картинок Моти
\usepackage{misccorr}
\usepackage{lscape}
\usepackage{cmap}


\usepackage{graphicx,xcolor}
\graphicspath{{Pictures/}}
\DeclareGraphicsExtensions{.pdf,.png,.jpg}

%----------------------------------------
%Список окружений
%----------------------------------------
\newenvironment {theor}[2]
{\smallskip \par \textbf{#1.} \textit{#2}  \par $\blacktriangleleft$}
{\flushright{$\blacktriangleright$} \medskip \par} %лемма/теорема с доказательством
\newenvironment {proofn}
{\par $\blacktriangleleft$}
{$\blacktriangleright$ \par} %доказательство
%----------------------------------------
%Список команд
%----------------------------------------
\newcommand{\grad}
{\mathop{\mathrm{grad}}\nolimits} %градиент

\newcommand{\diver}
{\mathop{\mathrm{div}}\nolimits} %дивергенция

\newcommand{\Def}[1]
{\underline{\textbf{#1}}} %определение

\newcommand{\RN}[1]
{\MakeUppercase{\romannumeral #1}} %римские цифры

\newcommand {\theornp}[2]
{\textbf{#1.} \textit{ #2} \par} %Написание леммы/теоремы без доказательства

\newcommand{\qrq}
{\ensuremath{\quad \Rightarrow \quad}} %Человеческий знак следствия

\newcommand{\qlrq}
{\ensuremath{\quad \Leftrightarrow \quad}} %Человеческий знак равносильности

\renewcommand{\phi}{\varphi} %Нормальный знак фи

\newcommand{\me}
{\ensuremath{\mathbb{E}}}

\newcommand{\md}
{\ensuremath{\mathbb{D}}}



%\renewcommand{\vec}{\overline}




%----------------------------------------
%Разметка листа
%----------------------------------------
\geometry{top = 3cm}
\geometry{bottom = 2cm}
\geometry{left = 3cm}
\geometry{right = 3cm}
%----------------------------------------
%Колонтитулы
%----------------------------------------
%\pagestyle{fancy}%Создание колонтитулов
%\fancyhead{}
%\fancyfoot{}
%\fancyhead[R]{\textsc{ОДУ: Контрольная работа 1}}%Вставить колонтитул сюда
%----------------------------------------
%Интерлиньяж (расстояния между строчками)
%----------------------------------------
%\onehalfspacing -- интерлиньяж 1.5
%\doublespacing -- интерлиньяж 2
%----------------------------------------
%Настройка гиперссылок
%----------------------------------------
\hypersetup{				% Гиперссылки
	unicode=true,           % русские буквы в раздела PDF
	pdftitle={Заголовок},   % Заголовок
	pdfauthor={Автор},      % Автор
	pdfsubject={Тема},      % Тема
	pdfcreator={Создатель}, % Создатель
	pdfproducer={Производитель}, % Производитель
	pdfkeywords={keyword1} {key2} {key3}, % Ключевые слова
	colorlinks=true,       	% false: ссылки в рамках; true: цветные ссылки
	linkcolor=blue,          % внутренние ссылки
	citecolor=blue,        % на библиографию
	filecolor=magenta,      % на файлы
	urlcolor=cyan           % на URL
}
%----------------------------------------
%Работа с библиографией 
%----------------------------------------
\renewcommand{\refname}{Список литературы}%Изменение названия списка литературы для article
%\renewcommand{\bibname}{Список литературы}%Изменение названия списка литературы для book и report
%----------------------------------------
\begin{document}



\begin{center}
\textbf{II контрольная работа по курсу "Обыкновенные \\
дифференциальные уравнения"
}\\
{Вариант I}
\end{center}
1. Для системы уравнений
$$
\begin{aligned}
&\dot{x}=5 x-2 y, \quad x(\cdot) \in \mathbb{R}, \; y(\cdot) \in \mathbb{R}, \\
&\dot{y}=4 x- y .
\end{aligned}
$$
\begin{itemize}
\item [(i)][1] определить все стационарные точки.
\item [(ii)][4] найти каноническое преобразование координат, выписать систему в новых координатах и изобразить ее фазовый портрет в этих координатах,
\item [(iii)][4] найти прямые в исходных координатах, проходящие через начало координат, при пересечении которых фазовые траектории параллельны осям $x$ и $y$.
\item [(iv)][2]  изобразить фазовый портрет системы в исходных координатах,
\item [(v)][1]  определить характер стационарных точек.
\end{itemize}
2. Для системы уравнений
$$
\begin{aligned}
&\dot{x}=x+2y+2e^{-t}, \quad x(\cdot) \in \mathbb{R},\; y(\cdot) \in \mathbb{R}, \\
&\dot{y}=2x+y
\end{aligned}
$$
\begin{itemize}
\item [(i)][3] найти общее решение соответствующей однородной системы.
\item [(ii)][3] найти траекторию неоднородной (исходной) системы, проходящую через начало координат,
\item [(iii)][2] для траектории, найденной в п. (ii), определить $\lim _{t \rightarrow \pm \infty} y(t) / x(t)$ (если они существуют).
\end{itemize}
3. Для дифференциального уравнения
$$
y''+4y'+4y=\frac{e^{-2x}}{2x^2}, \quad y(\cdot) \in \mathbb{R},\; x > 0
$$
\begin{itemize}
\item [(i)][2]  найти общее решение соответствующего однородного уравнения.
\item [(ii)][5]  найти общее решение неоднородного (исходного) уравнения.
\end{itemize}
4. Для дифференциального уравнения

$$
y''+y'-2y=3xe^x, \quad y(\cdot) \in \mathbb{R}
$$
\begin{itemize}
\item [(i)][2]  найти общее решение соответствующего однородного уравнения.
\item [(ii)][5] найти общее решение неоднородного (исходного) уравнения.

\end{itemize}





\newpage
\begin{center}
\textbf{II контрольная работа по курсу "Обыкновенные \\
дифференциальные уравнения"
}\\
{Вариант II}
\end{center}
1. Для системы уравнений
$$
\begin{aligned}
&\dot{x}=7 x+3 y, \quad x(\cdot) \in \mathbb{R}, \; y(\cdot) \in \mathbb{R}, \\
&\dot{y}= x- y .
\end{aligned}
$$
\begin{itemize}
\item [(i)][1] определить все стационарные точки.
\item [(ii)][4] найти каноническое преобразование координат, выписать систему в новых координатах и изобразить ее фазовый портрет в этих координатах,
\item [(iii)][4] найти прямые в исходных координатах, проходящие через начало координат, при пересечении которых фазовые траектории параллельны осям $x$ и $y$.
\item [(iv)][2]  изобразить фазовый портрет системы в исходных координатах,
\item [(v)][1]  определить характер стационарных точек.
\end{itemize}
2. Для системы уравнений
$$
\begin{aligned}
&\dot{x}=-x+4y+e^{-t}, \quad x(\cdot) \in \mathbb{R},\; y(\cdot) \in \mathbb{R}, \\
&\dot{y}=2x+y+1
\end{aligned}
$$
\begin{itemize}
\item [(i)][3] найти общее решение соответствующей однородной системы.
\item [(ii)][3] найти траекторию неоднородной (исходной) системы, проходящую через начало координат,
\item [(iii)][2] для траектории, найденной в п. (ii), определить $\lim _{t \rightarrow \pm \infty} y(t) / x(t)$ (если они существуют).
\end{itemize}
3. Для дифференциального уравнения
$$
y''+3y'+2y=\frac {1}{e^x+1}, \quad y(\cdot) \in \mathbb{R},\; x(\cdot) \in \mathbb{R},\\
$$
\begin{itemize}
\item [(i)][2]  найти общее решение соответствующего однородного уравнения.
\item [(ii)][5]  найти общее решение неоднородного (исходного) уравнения.
\end{itemize}
4. Для дифференциального уравнения

$$
y''-5y'+4y=4x^2 e^{2x}, \quad y(\cdot) \in \mathbb{R}
$$
\begin{itemize}
\item [(i)][2]  найти общее решение соответствующего однородного уравнения.
\item [(ii)][5] найти общее решение неоднородного (исходного) уравнения.

\end{itemize}






\newpage
\begin{center}
\textbf{II контрольная работа по курсу "Обыкновенные \\
дифференциальные уравнения"
}\\
{Вариант III}
\end{center}
1. Для системы уравнений
$$
\begin{aligned}
&\dot{x}=2 x+7 y, \quad x(\cdot) \in \mathbb{R}, \; y(\cdot) \in \mathbb{R}, \\
&\dot{y}= -2x- 2y .
\end{aligned}
$$
\begin{itemize}
\item [(i)][1] определить все стационарные точки.
\item [(ii)][4] найти каноническое преобразование координат, выписать систему в новых координатах и изобразить ее фазовый портрет в этих координатах,
\item [(iii)][4] найти прямые в исходных координатах, проходящие через начало координат, при пересечении которых фазовые траектории параллельны осям $x$ и $y$.
\item [(iv)][2]  изобразить фазовый портрет системы в исходных координатах,
\item [(v)][1]  определить характер стационарных точек.
\end{itemize}
2. Для системы уравнений
$$
\begin{aligned}
&\dot{x}=-x+4y+\sin t , \quad x(\cdot) \in \mathbb{R},\; y(\cdot) \in \mathbb{R}, \\
&\dot{y}=2x-y
\end{aligned}
$$
\begin{itemize}
\item [(i)][3] найти общее решение соответствующей однородной системы.
\item [(ii)][3] найти траекторию неоднородной (исходной) системы, проходящую через начало координат,
\item [(iii)][2] для траектории, найденной в п. (ii), определить $\lim _{t \rightarrow \pm \infty} y(t) / x(t)$ (если они существуют).
\end{itemize}
3. Для дифференциального уравнения
$$
y''+y=\frac {1}{\sin{x}}, \quad y(\cdot) \in \mathbb{R},\; x \in(0, \pi / 2)
$$
\begin{itemize}
\item [(i)][2]  найти общее решение соответствующего однородного уравнения.
\item [(ii)][5]  найти общее решение неоднородного (исходного) уравнения.
\end{itemize}
4. Для дифференциального уравнения

$$
y''+y=4xe^x, \quad y(\cdot) \in \mathbb{R}
$$
\begin{itemize}
\item [(i)][2]  найти общее решение соответствующего однородного уравнения.
\item [(ii)][5] ннайти общее решение неоднородного (исходного) уравнения.

\end{itemize}








\end{document}